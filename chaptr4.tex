\chapter{کارهای بعد از نصب}
\section{بررسی نکات انتشار}
 اوبونتو در هر نسخه پیشرفت‌های بسیاری می‌کند. آیا با پیشرفت‌های اوبونتوی ۱۳/۱۰ آشنا هستید؟ همین الان نکات انتشار اوبونتو را مطالعه کنید.

\section{نصب درایورها}
اگر از یک کاربر ویندوز بپرسید بعد از نصب ویندوز، نوبت چیست، بدون شک جواب خواهد داد: «نصب درایور»! اکثر کاربرانی که از ویندوز به سمت اوبونتو کشیده می‌شوند، در اوایل به فکر دانلود و نصب درایورها هستند!\\
در اوبونتو عمدتاً نیاز به نصب درایور خارجی ندارید و این سیستم عامل، بیش‌تر درایوهای مورد نیاز را به همراه دارد. اوبونتو را به صورت زنده بوت کنید و اگر همه چیز کار می‌کرد (صدا داشتید و صفحات وب را به خوبی توانستید مرور کنید)، آن را نصب کنید.\\
فقط ممکن است این احتمال وجود داشته باشد که اوبونتو بعضی از سخت‌افزارها، مثل کارت شبکه بی‌سیم را شناسایی نکند یا برای کارایی بیش‌تر گرافیکی، نیاز به نصب درایورهای انحصاری باشد.\\
برای نصب این درایورها، از \lr{System Settings}، گزینهٔ \lr{Software Sources} را انتخاب کنید و روی تب \lr{Additional Drivers} کلیک کنید. این ابزار سعی می‌کند برای سخت‌افزارهایی که شناخته نشده‌اند یا درایور بهتری برای‌شان موجود است، از اینترنت درایور را دانلود و سپس نصب کند.

\section[نصب کدک‌های چند رسانه‌ای، Flash Adobe و فونت‌های مناسب فارسی]{نصب کدک‌های چند رسانه‌ای، \lr{Adobe Flash} و فونت‌های مناسب فارسی}
اوبونتو بسیاری از کدک‌ها صوتی و تصویری معروف مثل \lr{MP3}، برنامه فلش و \ldots را به همراه ندارد. برای نصب آن‌ها، در مرکز نرم‌افزار (\lr{Software Center}) دنبال \lr{ubuntu-restricted-extras} بگردید و این بسته را نصب کنید.

\section{افزودن کیبورد فارسی}
برای این‌که بتوانید فارسی تایپ کنید، به این آدرس بروید:
\begin{flushleft}
\lr{System Settings -> Keyboard Layout -> Layouts}\\
\end{flushleft}
و با زدن دکمهٔ \lr{+} و انتخاب \lr{Persian}، زبان فارسی را به کیبوردتان اضافه کنید. با زدن دکمهٔ \lr{Options} در همان صفحه، مشخص کنید که با چه ترکیب دکمه‌ای، زبان ورودی عوض شود.
\section{نصب برنامه‌های اضافی}
برنامه‌های همراه اوبونتو زیاد هستند، اما برای تمامی کارهای روزانه کفایت نمی‌کنند. صدها برنامهٔ آزاد و غیر آزاد وجود دارد که به راحتی یک کلیک از مرکز نرم‌افزار نصب می‌شوند. لیست زیر چند برنامهٔ پیشنهادی را معرفی می‌کند.
\begin{itemize}
\item \lr{Chromium}: مرورگر سریع کرومیوم
\item \lr{Gimp}: ابزاری قوی برای ویرایش و ساخت تصاویر پیکلسی؛ معادل \lr{Adobe Photoshop}
\item \lr{LibreCAD}: ابزار طراحی نقشه‌های ساختمانی؛ معادل \lr{AutoCAD}
\item \lr{GNU Octave}: ابزاری عالی برای رایانش عددی و تجسم داده؛ معادلی برای \lr{MATLAB}
\end{itemize}

\section{فعال‌کردن راست به چپ در لیبره‌آفیس}
برای این کار، به این آدرس بروید:\\
\begin{flushleft}
\lr{Tools -> Options -> Language Settings -> Langauges}
\end{flushleft}
تیک گزینه \lr{Enable for complex text layout} را بزنید و از گزینهٔ \lr{CTL} که فعال می‌شود، \lr{Farsi} را برگزینید.

\section{استفاده از میزکار متفاوت}
اوبونتو به همراه میزکار \lr{Unity} عرضه می‌شود، ولی آن را به شما تحمیل نمی‌کند. برای داشتن میزکار \lr{KDE}، بستهٔ \lr{kubuntu-desktop} و برای داشتن \lr{Gnome Shell}، بستهٔ \lr{gnome-shell} را نصب کنید.

\section{سفارشی‌سازی میزکار}
جدا از این‌که چه میزکاری استفاده می‌کنید، می‌شود آن را با مجموعهٔ آیکون، فونت و پوسته‌های مختلف شخصی‌سازی کرد. مجموعه‌ای از بهترین آیکون و پوسته‌ها را از \lr{gnome-look.org} و \lr{kde-look.org} بگیرید. راهنمای استفاده هم همان جا وجود دارد.
\lr{Thunderbird}

\section{راه‌اندازی کلاینت ایمیل}
اوبونتو \lr{Thunderbird} را به همراه دارد که ابزاری برای مدیریت ایمیل‌هاست. بعد از اجرای تاندربرد، اسم، آدرس ایمیل و پسورد حساب ایمیل‌تان را به آن بدهید تا ایمیل‌هایتان را دریافت کند.

\section{همکاری در جامعهٔ کاربری اوبونتو}
اوبونتو با همکاری جامعهٔ کاربری‌اش زنده است و کتابی هم که می‌خوانید، با همکاری همین جامعهٔ کاربری ساخته شده است. تا جای ممکن، جامعهٔ کاربری را فراموش نکنید و به آن کمک کنید. وب‌سایت فارسی اوبونتو، جای خوبی برای شروع است. تعداد بسیار زیادی سوال در انجمن بدون پاسخ مانده‌اند و ده‌ها مدخل در ویکی وجود دارد که نیازمند به‌روزرسانی است.

\section{معرفی اوبونتو به دوستان و آشنایان}
اوبونتو را به دوستان و همکاران‌تان معرفی کنید تا آن‌ها هم با این سیستم عامل فوق‌العاده و آزاد آشنا بشوند.
