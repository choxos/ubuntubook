\chapter{کار با ترمینال}
از زیرمجموعه‌های لینوکس، رابط‌های گرافیکی یا \lr{GUI}ها هستند (\lr{Graphical User Interface}) که شما در آن می‌توانید موس‌تان را تکان دهید، کلیک کنید و بکشید، و می‌توانید بدون این‌که مستندات زیادی را بخوانید، کارهای‌تان را انجام دهید. محیط سنتی \lr{Unix}، یک رابط خط فرمان یا \lr{CLI} است (\lr{Command Line Interface}) که دستورات را در آن تایپ می‌کنید تا به کامپیوتر بگویید که چه کاری انجام دهد. این روش، خیلی سریع‌تر و قدرتمندتر است؛ اما لازم است که دستورات را بشناسید. در برخی شرایط، مخصوصا هنگام پیکربندی سیستم، مجبوریم که از ترمینال استفاده کنیم.

\section{آشنایی اولیه با ترمینال}
برای بازکردن ترمینال در اوبونتو، کافی است که روی لانچر کلیک کنید و چند حرف از کلمهٔ \lr{Terminal} را تایپ کنید تا آیکون ترمینال ظاهر شود. روی آن کلیک کنید. پنجرهٔ ترمینال باز خواهد شد.\\
در پنجرهٔ بازشده، یک خط مثل \lr{\texttt{ahmad@ahmad-netbook:$\sim$\$}} را مشاهده می‌کنید. \lr{\texttt{ahmad}} نام کاربری کنونی‌تان، \lr{\texttt{ahmad-netbook}} نام رایانهٔ‌تان، \lr{\texttt{$\sim$}} محل پوشهٔ کنونی‌تان (که این علامت، به معنی پوشهٔ خانگی‌تان است) و \lr{\texttt{\$}} هم به معنی دارابودن مجوز عادی و نداشتن مجوز کاربر ریشه است.\\

\section[،sudo اجرای دستورات با بالاترین مجوز دسترسی]{\lr{sudo}، اجرای دستورات با بالاترین مجوز دسترسی}
بعضی از دستورات، به اضافه‌کردن دستور \lr{\texttt{sudo}} (\lr{Super User Do}) به اولشان نیاز دارند. این در صورتی است که با فایل‌ها و پوشه‌هایی کار کنید که متعلق به حساب کاربری شما نباشد. این یک دستور ویژه است که به صورت موقت، به شما اجازهٔ تغییر تنظیمات کامپیوتر را می‌دهد. پس از واردکردن این دستور، ترمینال از شما گذرواژه را خواهد پرسید.

\subsection[تفاوت sudo با su]{تفاوت \lr{sudo} با \lr{su}}
در بسیاری از گنو/لینوکس‌های دیگر، امکان استفاده از دستور \lr{\texttt{sudo}} به صورت پیش‌فرض وجود ندارد. در این توزیع‌ها، به جای \lr{\texttt{sudo}}، از \lr{\texttt{su}} استفاده می‌شود.\\
\lr{\texttt{su}} مخفف عبارت \lr{Substitute User} به معنای «تغییر کاربر» است. یعنی علاوه بر تغییر کاربر کنونی به کاربر ریشه (کاربر ریشه یا \lr{root}، دارای بالاترین مجوز در سیستم‌های یونیکسی است)، می‌توان با واردکردن دستور \lr{\texttt{su \emph{user}}} (که به جای \lr{\texttt{\emph{user}}}، باید نام کاربر مورد نظر را بنویسید)، به عنوان آن کاربر فعالیت کرد. دستور \lr{\texttt{su}} هم وارد حساب کاربری ریشه خواهد شد. با زدن این دستور، خط ترمینال شبیه \lr{\texttt{root@ahmad-netbook:/home/ahmad\#}} خواهد شد. علامت \lr{\texttt{\#}}، نشان‌دهندهٔ حضور در حساب کاربری ریشه است.\\
در اوبونتو، امکان استفاده از \lr{\texttt{su}} هم وجود دارد. می‌توان با دستور \lr{\texttt{sudo su}}، وارد حساب کاربری ریشه شد. برای فعال‌کردن \lr{\texttt{su}}، باید ابتدا برای کاربر ریشه، گذرواژه‌ای را با دستور \lr{\texttt{sudo passwd root}} تعریف کنید. سپس می‌توانید با زدن \lr{\texttt{su}} و واردکردن گذرواژهٔ ریشه، بالاترین مجوزها را داشته باشید.\\
استفاده از دستورهای \lr{\texttt{su}} و \lr{\texttt{sudo su}} به هیچ وجه برای افراد تازه‌کار توصیه نمی‌شود. با داشتن مجوز ریشه و با زدن دستورهای نابه‌جا، امکان از بین رفتن اطلاعات و تنظیمات‌تان وجود دارد.\\
برای خارج‌شدن از ترمینال کاربر، کلمهٔ \lr{\texttt{exit}} را وارد کنید.

\section{دستورهای پرکاربرد ترمینال}
\subsection{دستورهای مربوط به کار با پرونده‌ها و پوشه‌ها}
\begin{itemize}
\item \textbf{\texttt{\Large pwd}}: این دستور به شما این امکان را می‌دهد که بدانید درچه پوشه‌ای هستید (\lr{pwd} مخفف عبارت \lr{Print Working Directory} است).  این اطلاعات را در نوار عنوان پنجره هم نشان داده می‌شود.

\item \textbf{\texttt{\Large ls}}: دستور \lr{\texttt{ls}} به شما پرونده‌های درون پوشه‌ای را که در آن هستید، نشان می‌دهد که اگر با بعضی انتخاب‌های دیگر (\lr{Options}) به کار رود، می‌تواند حجم پرونده‌ها، زمان و مکان ساخته‌شدن و مجوز دسترسی آن‌ها را مشاهده کنید. مثلاً \lr{\texttt{ls $\sim$}}، به شما پرونده‌های درون پوشهٔ \lr{home}تان را نشان می‌دهد.

\item \textbf{\texttt{\Large cd}}: دستور \lr{\texttt{cd}}، به شما اجازهٔ عوض‌کردن پوشهٔ کنونی را می‌دهد. هنگامی که یک ترمینال را باز می‌کنید، شما در پوشهٔ \lr{home}تان هستید. برای جابه‌جایی میان پوشه‌های سیستم، دستور \lr{\texttt{cd}} را به کار ببرد.\\
برای عقب‌رفتن به اندازهٔ یک پوشه، از \lr{\texttt{cd ..}} و برای برگشت به پوشهٔ پیشین، از \lr{\texttt{cd -}} استفاده کنید.

\item \textbf{\texttt{\Large cp}}: دستور \lr{\texttt{cp}}، یک رونوشت از پرونده را برای شما می‌سازد. برای مثال، \lr{\texttt{cp file foo}} یک کپی دقیق از \lr{\texttt{file}} را می‌سازد و نام آن را به \lr{\texttt{foo}} تغییر می‌دهد، اما پروندهٔ \lr{\texttt{file}} هنوز در محل خودش قرار دارد. اگر می خواهید از یک پوشه، کپی‌ای داشته باشید، باید از دستور \lr{\texttt{cp -r directory foo}} استفاده کنید.

\item \textbf{\texttt{\Large mv}}: دستور \lr{\texttt{mv}}، یک فایل را به مکانی دیگر منتقل می کند یا نام آن را تغییر می‌دهد. دستور \lr{\texttt{mv file foo}}، نام فایل \lr{\texttt{file}} را به \lr{\texttt{foo}} تغییر می دهد. \lr{\texttt{mv foo ~/Desktop}} فایل \lr{\texttt{foo}} را به پوشهٔ دسکتاپ شما منتقل می‌کند، اما نام آن را تغییر نمی‌دهد.

\item \textbf{\texttt{\Large rm}}: این دستور برای حذف‌کردن و برداشتن فایل‌ها به کار می‌رود. با قراردادن آپشن \lr{\texttt{-r}}، مانند \lr{\texttt{rm -r ~/Desktop/1/}}، می‌توان دستور را برای حذف پوشه‌ها هم به کار برد.

\item \textbf{\texttt{\Large mkdir}}: دستور \lr{\texttt{mkdir}} به شما اجازهٔ ساخت پوشه را می‌دهد. مثلاً \lr{\texttt{mkdir Music}} یک پوشه به نام \lr{\texttt{Music}} را خواهد ساخت.

\item \textbf{\texttt{\Large grep}}:
\end{itemize}

\subsection{دستورهایی برای آگاهی از اطلاعات سیستم}
\begin{itemize}
\item \textbf{\texttt{\Large df}}:
\item \textbf{\texttt{\Large du}}:
\item \textbf{\texttt{\Large free}}:
\item \textbf{\texttt{\Large top}}:
\item \textbf{\texttt{\Large uname}}:
\item \textbf{\texttt{\Large ifconfig}}:
\end{itemize}

\subsection{انتخاب‌ها}

\section[دستور man و به‌دست‌آوردن راهنمای دستورها]{دستور \lr{man} و به‌دست‌آوردن راهنمای دستورها}
\subsection[جست‌وجوی فایل‌های man]{جست‌وجوی فایل‌های \lr{man}}

\section{کلیدهای کاربردی در ترمینال}
