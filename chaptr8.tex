\chapter{بازی‌های گنو/لینوکس}
گرچه اساساً گنو/لینوکس برای بازی‌کردن به وجود نیامده است، اما به دلیل محبوبیت روزافزون آن میان کاربران غیرحرفه‌ای در سال‌های اخیر، بازی‌های خوبی برای گنو/لینوکس ارائه شده‌اند. هرچند که این بازی‌ها به هیچ عنوان در حد بازی‌های باکیفیت ویندوزی نیستند، اما با توجه به رایگان‌بودن (و در مواردی آزادبودن) آن‌ها قابل قبول‌اند و ارزش امتحان‌کردن دارند. در ادامه بازی‌هایی از انواع سبک‌ها معرفی می‌شوند.

\section{بازی‌های اول شخص}
\subsection[Nexuiz]{\lr{Nexuiz}}

\subsection[Tremulous]{\lr{Tremulous}}
\subsection[Terror Urban]{\lr{Urban Terror}}
\subsection[Warsow]{\lr{Warsow}}
\subsection[OpenArena]{\lr{OpenArena}}

\section{بازی‌های استراتژیک}
\subsection[Freeciv]{\lr{Freeciv}}
\subsection[2100 Warzone]{\lr{Warzone 2100}}
\subsection[OpenTTD]{\lr{OpenTTD}}
\subsection[FreeCol]{\lr{FreeCol}}

\section{بازی‌های مسابقه‌ای}
\subsection[TORCS]{\lr{TORCS}}
\subsection[Racer Tux]{\lr{Tux Racer}}

\section{بازی‌های نقش‌آفرینی}
\subsection[Auteria]{\lr{Auteria}}
\subsection[PlaneShift]{\lr{PlaneShift}}


\section{بازی‌های آموزشی}
\subsection[gbrainy]{\lr{gbrainy}}

\section[بازی‌های Steam]{بازی‌های \lr{Steam}}
