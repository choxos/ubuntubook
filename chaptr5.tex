\chapter{نرم‌افزارهای برتر اوبونتو}
در این بخش، به معرفی برترین و کاربردی‌ترین نرم‌افزارهای اوبونتو می‌پردازیم و توضیح مختصری راجع به هر یک از نرم‌افزارها ارایه می‌کنیم. لازم به ذکر است که تمامی نرم‌افزارهای زیر، آزاد، متن‌باز و رایگان بوده و شما می‌توایند این نرم‌افزارها را به راحتی و با جست‌و‌جو در \lr{Software Center} نصب کنید.

\section[LibreOffice]{\lr{LibreOffice}}
لیبره آفیس یکی از اولین نیازمندی‌های کاربر متوسط است. این بستهٔ نرم‌افزاری، جایگزین مناسبی برای نرم‌افزار آفیس مایکروسافت است و از بخش‌های زیر تشکیل می‌شود:
\begin{itemize}
\item \lr{\textbf{Writer}}: برنامه‌ای است برای نوشتن و ویرایش متن. این نرم‌افزار، زبان فارسی را کاملاً پشتیبانی می‌کند. خروجی پیش‌فرض آن، \lr{odt} است اما می‌توانید خروجی هایی مانند \lr{doc} و \lr{pdf} نیز داشته باشید.
\item \lr{\textbf{Impress}}: نرم‌افزار ساخت فایل‌های ارائه که معادل \lr{PowerPoint} در مجموعهٔ آفیس مایکروسافت است.
\item \lr{\textbf{Calc}}: این نرم‌افزار برای ساخت و ویرایش فایل‌های صفحه‌گسترده است.
\item \lr{\textbf{Draw}}: برای طراحی های ساده گرافیکی مورد استفاده قرار می گیرد.
\item \lr{\textbf{Base}}: نرم افزاری برای طراحی مفهومی پایگاه‌داده و روابط بین جداول است و عملکردی مانند \lr{MS Access} و \lr{Power Designer} دارد.
\item \lr{\textbf{Math}}: کار نوشتن فرمول‌های ریاضی را انجام می‌دهد.
\end{itemize}

\section[Gimp]{\lr{Gimp}}
نرم‌افزاری است که برای طراحی‌های گرافیکی و ویرایش تصاویر استفاده می‌شود و تا حدودی شبیه \lr{Photoshop} است. از فایل‌های \lr{psd} نیز پشتیبانی می‌کند.\lr{Gimp} ابزار و فیلترهای متنوعی برای ویرایش تصاویر دارد که به ساخت تصاویری زیبا کمک می‌کند.

\section[Inkscape]{\lr{Inkscape}}
یکی از حرفه‌ای ترین نرم‌افزارها در زمینه طراحی تصاویر بُرداری (\lr{vector}) است. بسیاری از طرح‌ها و آیکن‌های موجود در اوبونتو، با این نرم‌افزار طراحی شده‌اند. \lr{Inkscape} جایگزین مناسبی برای نرم‌افزار \lr{illustrator} به حساب می‌آید.

\section[Blender]{\lr{Blender}}
این نرم‌افزار، برای تمامی طراحان سه‌بعدی دنیای کامپیوتر، توصیه می‌شود. \lr{Blender} نرم‌افزاری است که در بسیاری از فیلم‌های هالیوودی و بازی‌های کامپیوتری معروف استفاده شده است و همچنین انیمیشن‌های زیادی با این نرم‌افزار ساخته شده است.

\section[K3b]{\lr{K3b}}
نرم‌افزاری برای کپی‌برداری از \lr{CD} و \lr{DVD} است. \lr{K3b} بدون شک یکی از بهترین نرم‌افزارهای موجود برای این کار است.

\section[Darktable]{\lr{Darktable}}
چه یک عکاس حرفه‌ای باشید، چه یک کاربر سادهٔ کامپیوتر، با عکس سر و کار خواهید داشت. مهم نیست این عکس‌ها با دوربین حرفه‌ای گرفته می‌شوند یا دوربین تلفن‌همراه‌تان، مهم نیست که این عکس‌ها از دل طبیعت گرفته شده‌اند یا عکس‌هایی خانوادگی هستند؛ تمامی این عکس‌ها احتیاج به مدیریت و ویرایش در میزان رنگ و روشنی تصویر یا تغییراتی از این دست دارند. \lr{Darktable} تمامی چنین نیازهایی را پاسخ خواهد داد.

\section[Virtualbox]{\lr{Virtualbox}}
با کمک این نرم‌افزار شما قادر خواهید بود تا در اوبونتو، سیستم عامل دیگری مانند ویندوز را نصب کنید و با اختصاص منابع سیستمی به آن، می‌توانید کاملاً از آن سیستم عامل و نرم‌افزارهایی که روی آن نصب کرده اید، استفاده کنید.

\section[Wine]{\lr{Wine}}
معمولاً در اوایل دوران کوچ به سیستم عامل دیگر، زمان‌هایی پیش می‌آید که به نرم‌افزارهای سیستم عامل قبلی خود نیاز پیدا کنید و به دلیل آشنا نبودن با نرم‌افزارهای جایگزین موجود، شاید در ابتدای کار با سیستم عامل جدید کمی آزار‌دهنده باشد. \lr{Wine} نرم‌افزاری است که به شما امکان اجرای بسیاری از نرم‌افزارها و بازی‌های سیستم عامل ویندوز را روی اوبونتو می‌دهد.

\section[Goldendict]{\lr{Goldendict}}
وجود یک لغت‌نامه در رایانه، نیازی است که کاربران کم‌سن‌وسال تا استادان زبان را شامل می‌شود. \lr{Goldendict}، یک برنامه تمام‌عیار برای این نیاز است. این برنامه، از کتاب‌خانه لغات \lr{Babylon} با قالب \lr{bgl} نیز پشتیبانی می‌کند.

\section[VLC]{\lr{VLC}}
شاید با \lr{VLC} در سیستم عامل‌های دیگر نیز کار کرده باشید. \lr{VLC} در زمینه پخش فایل‌های موسیقی و ویدیویی، همه‌فن‌حریف است و از بیش‌تر فرمت‌ها، از \lr{MP3} گرفته تا \lr{Bluray}، پشتیانی می‌کند.
\lr{Bluray}
