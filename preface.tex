\chapter{مقدمه}
\emph{\lr{Ubuntu}} تنها یک سیستم عامل آزاد و متن‌باز  با بیش از ۲۰ میلیون کاربر در سرتاسر جهان نیست؛ اوبونتو یک فرهنگ است، یک خلاقیت بزرگ، یک پروژهٔ گروهی، در نوبهٔ خود مهم‌ترین و برجسته‌ترین. اوبونتو یک جامعه از مردم است.\\
اگر در حال خواندن این راهنما هستید، ممکن است تصمیم گرفته باشید که از فضای سیستم عامل‌هایی مانند \lr{Windows} و \lr{Mac OS X} دور شوید و یا شاید اخیراً اوبونتو را بر روی رایانهٔ‌تان نصب کرده‌اید، اما مطمئن نیستید که از کجا باید شروع کنید.\\
استفاده از یک سیستم عامل جدید می‌تواند ترسناک باشد، مخصوصاً وقتی که با کلمه‌های ناآشنا روبه‌رو می‌شوید. بسیاری از مردم با اصطلاحات فنی یک سیستم عامل آشنا نیستند و معتقدند که این مفاهیم برایشان خیلی پیشرفته است. در واقع این موضوع درست نیست. اوبونتو به راحتی نصب می‌شود و استفاده از آن ساده است. و از همه مهم‌تر این‌که: \emph{کاملاً آزاد و  رایگان است}.\\

این راهنما برای کسانی است که به تازگی استفاده از گنو/لینوکس را شروع کرده‌اند و این امکان را به آن‌ها می‌دهد که تمام ابزارهای موردنیاز را بشناسند و از آن‌ها به درستی استفاده کنند.\\
شما با خواندن این کتاب می‌آموزید که چگونه کارهای زیر را انجام دهید:
\begin{itemize}
\item نصب و راه‌اندازی اوبونتو بر روی رایانه‌ٔتان
\item پشتیبانی فنی در این محیط
\item درک فلسفهٔ اوبونتو
\item ایجاد وحدت در رابط میز کاربری
\item استفاده از نرم افزارهای سازگار با اوبونتو
\end{itemize}
